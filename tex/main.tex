% Preamble
\documentclass[12pt]{article}

% Packages
\usepackage{amsmath}
\usepackage{hyperref}
\usepackage[utf8]{inputenc}
\usepackage[T1]{fontenc}

\title{\textbf{Zufallszahlen Spektraltest}}
\author{Christian Locatelli, Alexander Wallrodt}
\date{\today}

% Document
\begin{document}
    % Titelseite
    \maketitle
    \clearpage

    % Inhaltsverzeichnis
    \tableofcontents
    % Liste der Tabellen
    \listoftables
    % Liste der Bilder
    \listoffigures

    \clearpage


    \section{Einleitung}\label{sec:Einleitung}
    Im folgenden Projekt geht es um die Frage, inwieweit von einem Computer per Algorithmus erzeugte Pseudo-Zufallszahlen
    auch wirklich als zufällig zu Erachten sind.
    Wir werden daher insgesamt vier unterschiedliche Zufallszahlengeneratoren untersuchen und dabei schauen,
    ob zwischen den erzeugten Zufallszahlen ein kausaler Zusammenhang besteht,
    oder ob diese Generatoren den Zufall gut simulieren können.



    \section{Das Pythonscript}\label{sec:das-pythonscript}
    Wie ist das Pythonscript aufgebaut und wie funktioniert es?

    \subsection{Der LCG}\label{subsec:der-lcg}
    Der LCG ist ein einfacher Generator, welcher mit den Startwerten $a$, $x_n$, $c$, $m$ aufgerufen wird.
    $a$ ist der Faktor, $n$ der Startwert, $c$ ist das Inkrement und $m$ der Modulus.
    Diese können beliebig gewählt werden, jedoch nur bestimmte sehen auch zufällig aus.
    Für unser Projekt haben wir die Werte des TI-59 von Texas Instruments gewählt~\cite{lcg}.

    \begin{align*}
    x_n = (a * x_n + 1) \bmod m
    \end{align*}

    \subsection{Der Random.org Generator}\label{subsec:der-random.org-generator}
    Random.Org ist eine Internetseite, die sich darauf spezialisiert hat, Zufallszahlen zu erzeugen.
    Sie wird unter anderem für Lotterien und Online-Spiele, aber auch für diverse wissenschaftliche
    Anwendungen verwendet.
    Hierbei behaupten die Entwickler, der Generator würde mithilfe von atmosphärischem
    Rauschen echte Zufallszahlen erstellen, die besser sind als die von Computer erstellten Pseudo-Zufallszahlen~\cite{random-org}.

    \subsection{Die Zufallsgeneratoren von Python}\label{subsec:die-zufallsgeneratoren-von-python}
    Python hat selbst verschiedene Zufallszahlengeneratoren integriert, die Pseudo-Zufallszahlen erstellen können.
    Hierbei wird der sogenannte Mersenne-Twister verwendet, welcher floats in einer Genauigkeit von 53 Bit erzeugt und welcher
    eine Periode von $p=2^{19937}-1$ besitzt.
    Er gilt als einer der am häufigsten getesteten Zufallsgeneratoren, ist jedoch vollständig deterministisch,
    weswegen er beispielsweise für Kryptografie ungeeignet wäre~\cite{python-random,mersenne-twister}.
    Im Programm werden die Methodem random.random() und numpy.random() getestet,
    welche eine Fließkommazahl zwischen 0 und 1 erzeugen.



    \section{Ergebnisse}\label{sec:Ergebnisse}
    Um die Generatoren auf ihre Zufälligkeit zu überprüfen haben wir den Spektraltest genutzt.
    Bei diesem werden Tupel gebildet, man kann sich aussuchen wie hoch die Tupelanzahl ist.
    Zur Veranschaulichung haben wir 2er-Tupel gebildet, welche in einem XY-Diagramm, sowie 3er-Tupel welche in einem
    XYZ-Diagramm dargestellt werden können.

    \subsection{Durchgeführte Tests}\label{subsec:durchgeführte-tests}
    Wir haben den Spektraltest auf verschiedene Generatoren angewandt.
    Diese geben alle Fließkommazahlen zwischen 0 und 1 zurück, um einen guten Vergleich zu haben.
    Wir haben bei jedem Generator 100000 Zufallszahlen abgerufen,
    jedoch sind wir beschränkt bezüglich der Anzahl der Zufallszahlen auf 10000 bei Random.org.

    \begin{table}[h]

        \caption[Ergebnisstabelle]{Ergebnisstabelle}

        \centering

        \begin{tabular}{|c||c|c|}

            \hline
            & XY-Tupel & XYZ-Tupel \\

            \hline
            \hline
            LCG & Picture 1.1 & Picture 1.1 \\

            \hline
            Random Library & Picture 2.1 & Picture 2.2 \\

            \hline
            Numpy Library & Picture 3.1 & Picture 3.2 \\

            \hline
            Random.org & Picture 4.1 & Picture 4.2 \\

            \hline

        \end{tabular}\label{tab:ergebnisse}

    \end{table}

    \subsection{Fazit}\label{subsec:fazit}
    Die Ergebnisse zeigen, dass die Zufallsgeneratoren der Python-Bibliotheken zufällig aussehen, da wir jedoch wissen,
    dass die Generatoren der Bibliotheken auf dem Mersenne-Twister basieren,
    können wir diese als \textit{gute} Pseudozufallszahlengeneratoren bezeichnen.
    Erst mit sehr hohen Perioden wären dort Muster zu erkennen.
    Der LCG ist kein guter Generator, da in der 3D- und 2D-Darstellung mit leichtigkeit Muster zu erkennen sind.
    Der Random.org Generator zeigt ebenfalls keine Muster auf, demnach wären seine Werte echt zufällig,
    dies stimmt auch mit der Aussage von Random.org überein~\cite{random-org}.


    \vfill

    \begin{thebibliography}{12345}

        \bibitem{lcg}
        \url{https://de.wikipedia.org/wiki/Kongruenzgenerator}

        \bibitem{random-org}
        \url{https://www.random.org/analysis}

        \bibitem{python-random}
        \url{https://docs.python.org/3/library/random.html}

        \bibitem{mersenne-twister}
        \url{https://de.wikipedia.org/wiki/Mersenne-Twister}

    \end{thebibliography}

\end{document}