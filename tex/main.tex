% Preamble
\documentclass[12pt]{article}

% Packages
\usepackage{amsmath}
\usepackage{hyperref}
\usepackage[utf8]{inputenc}
\usepackage[T1]{fontenc}

\title{\testbf{Zufallszahlen Spektraltest}}
\author{Christian Locatelli, Alexander Wallrodt}
\date{\today}

% Document
\begin{document}
    % Titelseite
    \maketitle
    \clearpage

    % Inhaltsverzeichnis
    \tableofcontents
    % Liste der Tabellen
    \listoftables
    % Liste der Bilder
    \listoffigures

    \clearpage


    \section{Einleitung}\label{sec:Einleitung}
    Im folgenden Projekt geht es um die Frage, inwieweit von einem Computer per Algorithmus erzeugte Pseudo-Zufallszahlen
    auch wirklich als zufällig zu Erachten sind.
    Wir werden daher insgesamt vier unterschiedliche Zufallszahlengeneratoren untersuchen und dabei schauen,
    ob zwischen den erzeugten Zufallszahlen ein kausaler Zusammenhang besteht,
    oder ob diese Generatoren den Zufall gut simulieren können.



    \section{Das Pythonscript}\label{sec:das-pythonscript}
    Wie ist das Pythonscript aufgebaut und wie funktioniert es?

    \subsection{Der LCG}\label{subsec:der-lcg}
    Der LCG ist ein einfacher Generator, welcher mit den Startwerten $a$, $x_n$, $c$, $m$ aufgerufen wird.
    $a$ ist der Faktor, $n$ der Startwert, $c$ ist das Inkrement und $m$ der Modulus.
    Diese können beliebig gewählt werden, jedoch nur bestimmte sehen auch zufällig aus.
    Für unser Projekt haben wir die Werte des TI-59 von Texas Instruments gewählt~\cite{lcg}.

    \begin{align*}
    x_n = (a * x_n + 1) mod m
    \end{align*}

    \subsection{Der Random.org Generator}\label{subsec:der-random.org-generator}
    Random.Org ist eine Internetseite, die sich darauf spezialisiert hat, Zufallszahlen zu erzeugen.
    Sie wird unter anderem für Lotterien und Online-Spiele, aber auch für diverse wissenschaftliche
    Anwendungen verwendet.
    Hierbei behaupten die Entwickler, der Generator würde mithilfe von atmosphärischem
    Rauschen echte Zufallszahlen erstellen, die besser sind als die von Computer erstellten Pseudo-Zufallszahlen~\cite{random-org}.

    \subsection{Die Zufallsgeneratoren von Python}\label{subsec:die-zufallsgeneratoren-von-python}
    Python hat selbst verschiedene Zufallszahlengeneratoren integriert, die Pseudo-Zufallszahlen erstellen können.
    Hierbei wird der sogenannte Mersenne-Twister verwendet, welc







    \section{Ergebnisse}\label{sec:Ergebnisse}
    Um die Generatoren auf ihre Zufälligkeit zu überprüfen haben wir den Spektraltest genutzt.
    Bei diesem werden Tupel gebildet, man kann sich aussuchen wie hoch die Tupelanzahl ist.
    Zur Veranschaulichung haben wir 2er-Tupel gebildet, welche in einem XY-Diagramm, sowie 3er-Tupel welche in einem
    XYZ-Diagramm dargestellt werden können.

    \vfill

    \begin{thebibliography}{12345}

        \bibitem{lcg}
        \url{https://de.wikipedia.org/wiki/Kongruenzgenerator}

        \bibitem{random-org}
        \url{https://www.random.org/analysis/}

    \end{thebibliography}
\end{document}